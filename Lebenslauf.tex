\RequirePackage[l2tabu, orthodox]{nag}
\documentclass[11pt, a4paper]{moderncv}


\moderncvstyle{classic}
\moderncvcolor{grey}
%\moderncvicons{letters}
\usepackage{mathpazo}
\renewcommand{\sfdefault}{\rmdefault}
\usepackage[utf8]{inputenc}
\usepackage{moderntimeline}
\tltext{\tiny}
\usepackage[scale=0.81]{geometry}
%\usepackage[scale=0.85, paperheight=75cm, top=2cm, bottom=2cm]{geometry}
\setlength{\hintscolumnwidth}{3.3cm} % if you want to change the width of the column with the dates
\usepackage{microtype} % keep at the end of the imports

\name{Philipp}{Krenn}
\title{\href{https://xeraa.net}{https://xeraa.net}} % optional
\address{Billrothstraße 11/8}{A-1190, Wien, Österreich} % optional
\renewcommand{\mobilephonesymbol}{}
\phone[mobile]{+43 676 3428954} % optional
\renewcommand{\emailsymbol}{}
\email{pk@xeraa.net} % optional
\renewcommand{\faxphonesymbol}{}
\phone[fax]{\href{https://twitter.com/xeraa}{@xeraa}} % optional
\extrainfo{\href{https://github.com/xeraa}{https://github.com/xeraa}} % optional
\photo[84pt]{Picture} % optional
%\quote{Macher, Denker, Experte, Lernender.} % optional

\nopagenumbers{}


\begin{document}
\maketitle


\section{Persönliche Daten}
\cventry{Name}{Philipp Krenn, BSc}{}{}{}{}
\cventry{Geboren}{21. Oktober 1984}{}{}{}{}
\cventry{Staatsbürgerschaft}{Österreich}{}{}{}{}
\cventry{Familienstand}{Ledig}{}{}{}{}


\tlmaxdates{2003}{2015}


\section{Ausbildung}
\tlcventry{2010}{0}{Software Engineering \& Internet Computing \textnormal{und} Information \& Knowledge Management}{Masterstudium}{TU Wien}{voraussichtlicher Abschluss im Q2/2015}{}
\tlcventry{2004}{2010}{Software \& Information Engineering}{Bakkalaureatsstudium}{TU Wien}{}{}


\section{Berufstätigkeit}
\tlcventry{2013}{0}{Betrieb und Backend Entwickler}{ecosio GmbH}{}{}{Fortsetzung des TU Wien Projekts in einem Startup, verantwortlich für Operations (Amazon Web Services), Builds und Deployment (Maven, Jenkins), sowie Datenbanken (MongoDB, MySQL, Hazelcast)} %seit 10/2013
\tlcventry{2012}{0}{Trainer}{tutego~/ Christian Ullenboom}{in Deutschland und der Schweiz}{}{Schulungen für NoSQL (MongoDB, ElasticSearch, Redis, Hazelcast, CouchDB, Neo4j, Cassandra und HBase) und Cloud Computing (AWS).} %seit 08/2012
\tlcventry{2009}{0}{Trainer}{SPC GmbH}{}{}{Schulungen und Workshops für Java Entwicklung und Architektur, XML, MS Project und Projektmanagement.} %seit 05/2009
\tlcventry{2009}{2014}{Webentwickler}{Men on the Moon GmbH}{}{}{Entwicklung  von \href{http://www.contentaward.at}{http://www.contentaward.at} und \href{http://www.mediaquarter.at}{http://www.mediaquarter.at} mit SilverStripe, MySQL und jQuery.} %10/2009 bis 12/2014
\tlcventry{2010}{2013}{Wissenschaftlicher Mitarbeiter}{Ecommerce Gruppe}{TU Wien}{}{Evaluierung und Implementierung aktueller Datenbank-Technologien (RDBMS vs. NoSQL) und Cloud Lösungen für ein Java Industrieprojekt~-- \href{http://erpel.at}{http://erpel.at}.} %10/2010 bis 09/2014
\tlcventry{2010}{2012}{Consultant}{Splendit IT-Consulting GmbH}{}{}{Consultant für Software Qualitätssicherung, Analyse und Schulungen.} %10/2010 bis 12/2012
\tldatecventry{2012}{Mentor}{}{Google Summer of Code 2012}{}{"Content Personalization" für SilverStripe Ltd., Teilnehmer am Mentor Summit in Mountain View.} %05/2012 bis 09/2012
\tlcventry{2007}{2010}{Trainer}{Splendit IT-Consulting GmbH}{}{}{Kurse für JavaScript, jQuery, JSF, JSP, Design Patterns, Objektorientierung, JBoss AS, PostgreSQL, UML2 und WebServices bei Oracle GmbH, GNC GmbH und BTC Weiterbildung GmbH.} %11/2007 bis 09/2010
\tlcventry{2004}{2010}{Systemadministrator}{LMC Town to Town Ltd}{}{}{Betreuung von Windows SBS und Clients sowie Linux Firewall und Webserver.} %08/2004 bis 09/2010
\tlcventry{2009}{2010}{Tutor}{Distributed Systems Group}{TU Wien}{}{Master Lehrveranstaltungen \emph{Software Architekturen} und \emph{Entwurfsmethoden für verteilte Systeme}.} %10/2009 bis 06/2010
\tlcventry{2007}{2009}{Webentwickler}{Farthofer Gastronomie KEG}{}{}{Erstellung von \href{http:// www.kriterium.at}{http:// www.kriterium.at} mit Smarty und jQuery.} %02/2007 bis 12/2009
\tldatecventry{2009}{Webentwickler}{Heumarkt Veranstaltungs GmbH}{}{}{Entwicklung von \href{http://www.sandinthecity.at}{http://www.sandinthecity.at} mit Drupal, MySQL und jQuery.} %02/2009 bis 09/2009
\tlcventry{2008}{2009}{Trainer}{ppedv AG}{in Österreich und Deutschland}{}{Schulungen für JavaScript, MS Project und MS SQL Server.} %12/2008 bis 05/2009
\tldatecventry{2007}{Student Developer}{}{Google Summer of Code 2007}{}{"Support for Multiple Databases" für SilverStripe Ltd. implementiert in PHP~/ PDO für MySQL, PostgreSQL und MS SQL Server~-- Mentor Brian Calhoun (CEO).} %05/2007 bis 09/2007
\tlcventry{2006}{2007}{Webentwickler}{Splendit IT-Consulting GmbH}{Universitätsprojekt}{}{Entwicklung einer JEE Test-Applikation in einem Team von sechs Studenten.} %10/2006 bis 01/2007
\tldatecventry{2006}{Sommer-Praktikum}{Frequentis AG}{IT Second Level Support}{}{}


\section{Sprachen}
\cvline{Deutsch}{Muttersprache}
\cvline{Englisch}{Verhandlungssicher}
\cvline{Französisch}{Maturaniveau}


\section{Publikationen}
\tldatecventry{2011}{Buch}{SilverStripe 2.4 Module Extension, Themes and Widgets: Beginner's Guide}{Packt Publishing Limited}{ISBN 978-1849515009 }{Moderne Webentwicklung mit SilverStripe, einem PHP5 CMS und Framework.}
\tldatecventry{2003}{Forschungsarbeit}{Strategic Information Warfare in Cyberspace}{}{}{}


\section{Vorträge und Präsentationen}
\tldatecventry{2014}{MongoDB Replication — What Could Go Wrong}{}{Javantura}{Zagreb}{}{} %11/2014
\tldatecventry{2014}{Morphia: Painfree JPA for MongoDB}{}{Javantura}{Zagreb}{}{} %11/2014
\tldatecventry{2014}{nginx für SilverStripe}{}{SilverStripe Europa Konferenz}{Linz}{}{} %10/2014
\tlcventry{2013}{0}{Meetups}{}{Organisator von \href{http://www.meetup.com/ViennaDB-The-Austrian-Database-Meetup-Group/}{ViennaDB}, \href{http://www.meetup.com/Papers-We-Love-Vienna/}{Papers We Love Vienna} und \href{http://www.meetup.com/SilverStripe-Austria/}{SilverStripe Austria}, regelmäßig Vortragender bei anderen Meetups}{}{}{}
\tldatecventry{2011}{Open Source Entwicklungs-Plattformen}{}{Präsentation für Future Network bei Gentics GmbH}{}{}{} %11/2011
\tldatecventry{2011}{Morphia}{}{Vortrag bei der MongoUK2011}{London}{}{} %3/2011
\tldatecventry{2010}{Qualität von Open Source Software}{}{Präsentation für Future Network bei Microsoft Österreich GmbH}{}{}{} %11/2010
\tldatecventry{2010}{IBM Smart Talk}{}{Podiumsdiskussion über Cloud Computing an der TU Wien}{}{}{} %11/2010


\section{EDV Kenntnisse}
\subsection{Software Entwicklung}
\cvline{Java}{Java, JEE, Spring, JMS, Maven und Gradle, JUnit und TestNG, Jenkins, Sonar}
\cvline{PHP}{PHP, SilverStripe, Drupal, Smarty, TYPO3}
\cvline{Web-Technologien}{HTML, CSS, JavaScript, jQuery}
\cvline{Datenbanken}{MongoDB, MySQL, ElasticSearch, Redis, Hazelcast, PostgreSQL, Neo4j}
\cvline{Datenaustausch}{WebServices (SOAP, REST), JSON, XML (DTD, Schema, XPath, XSLT)}
\cvline{Tools}{Shell Scripts, Amazon Web Services (EC2, RDS, S3, VPC), UML2}
\cvline{Paradigmen}{Objektorientierte, funktionale und logikorientierte Programmierung}

\subsection{Betriebssysteme}
\cvline{*nix}{Mac OS X, Ubuntu, Debian}
\cvline{Microsoft Windows}{Server und Client}

\subsection{Applikationen}
\cvline{Server}{Ansible, nginx, Apache Tomcat, Jetty, ActiveMQ}
\cvline{Desktop}{Git, IntelliJ, Vagrant, VirtualBox, Vim, \LaTeX, Prezi, Microsoft Project und Visio}

\subsection{Zertifizierungen}
\tldatecventry{2013}{M101J: MongoDB for Java Developers}{10gen Inc}{}{}{}{}
\tldatecventry{2012}{M102: MongoDB for DBAs}{10gen Inc}{}{}{}{}


\tlmaxdates{2001}{2014}
\section{Zusatzqualifikationen}
\tlcventry{2009}{0}{Wiener Walzer Formation}{Tanzschule Elmayer}{}{}{}
\tlcventry{2001}{0}{Golf Mannschaft}{Wienerberg}{}{}{}


\end{document}